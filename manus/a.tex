\documentclass{article}
\usepackage{mathtools,natbib}

\begin{document}

\section{Grain rotation}
Grain rotation is described using a Jefferys-Type equation \citep{azuma94}:
\[
 \dot{c} = D \cdot c + c^T \cdot D \cdot c
\]
This causes c-axes to rotate away from extensional axes.
\section{Grain radius evolution}
Each grain in the simulation has an associated radius $r_i$. The grain growth rate is taken as a function of the driving force of grain growth. The driving force of each grain is calculated as if the grain is embedded in a homogenous medium with the average properties (i.e., elastic strain energy) of the fabric sample.

The assumptions are the following: Three-dimensional grains have a surface area proportional to $r^2$ and a volume proportional to $r^3$, where $r_i$ is the radius of a sphere with the same volume. Assuming a constant scaled shape between grains, then a polycrystal of grains $\{g_i}$ with volume $V=\sum v_i$ where $v_i$ is the volume of grain $i$, has surface area $S=\sum s_i$. For any three-dimensional shape, $\frac{\alpha}{r} v_i =  s_i$, for some constant $\alpha$. Thus, $S=\beta \sum r_i^2$, for some constant $\beta$.

Let $g^{\star}$ be an arbitrary interior grain. This grain has neighbors $g_1,...,g_m$, with radii $r_i$. $g$ has a surface area $T$. The surface $g^{\star}$ shares with neighbor $n_i$ is $S_i$, with surface area $s_i$. Now, assume that $s_i$ is proportional the the surface area of grain $g_i$, or equivalently, $s_i=\alpha r_i^2$.

Now, examine the interface between $g^{\star}$ and $n_i$. Due to grain growth, recrystallization, or other processes, this moves with velocity $v_i$ outwardly normal to the surface. Thus, taken over all neighboring grains, $\frac{dV^{\star}}{dt}=\sum_{i}s_i v_i$. Therefore, if a velocity between two grains can be determined, then this provides a means of determining grain size evolution based on the properties of a grain and its neighbors.

To implement this for a polycrystal, the less computationally intensive way is to declare that each grain has several neighbors with which to determine the growth rate. To deal with grains disappearing or nucleating, originally the polycrystal is declared to have a number of grains with zero radius. Nucleated grains simply assign a nonzero radius to a formerly zero radius grain, and consumed grains are assigned a zero radius. 

\section{Grain Growth}
Normal grain growth is induced by a driving force related to the local curvature at the grain surface

From \citet{durand2006}, the growth rate of a grain in the presence of Zener pinning is can be given by

\[\frac{dr}{dt} = K \left( \frac{1}{R}-\frac{P}{\alpha \gamma} \right)
\]

where $K$ is an Arrhenius factor dependent on temperature, and $P$ the counteracting force due to pinning. The difficulty in implementing this using the procedure outlined in the previous section is to determine the radius of curvature at the boundary between the two grains. Clearly, the curvature of each grain must be the negative of the other. In addition, taken over the entire boundary, the integral of the curvature over the boundary of the grain must always be $4 \pi$. This means that, for example, the grain having negative curvature must "make up" that curvature elsewhere on its boundary. 

A reasonable way to handle this (which I am taking)  is to take the curvature of the boundary to be the harmonic mean of the curvatures of the two surfaces (or the regular mean of the radii). This means that the smaller grain will have positive curvature on the boundary, which is a reasonable assumption. The other, far less practical, alternative is to solve a constrained optimization problem to determine the curvatures such that the curvature of all boundaries of each grain adds to $4 \pi$. For this to be well-posed, some grains must be declared as grains on the exterior of the polycrystal, with the only constraint being that their total is $4 \pi$ as well. 




\section{Recrystallization}
Migration (dynamic) recrystallization is primarily driven by lower strain energy (easy glide) grains consuming higher strain energy grains \citep{duval1995}. At least at larger scales (after the initial nucleation), this reduction in strain energy can be incorporated into the driving force. High dislocation densities from work-hardening in more highly plastically strained grains can also contribute to the driving force, as these dislocations are annihilated by the advancing boundary. The model currently calculates the effective stress on each grain following \citet{azuma96}, which depends on the bulk stress and the orientation of each grain and its neighbors. This stress is then converted into strain energy under the assumption of elasticity.

\section{Fabric model usage}
The fabric is specified as an instance of a datatype subtyping a general 'Fabric' containing most of the relevant physical information and model parameters, such as number of grains, strain rate, neighbor connectivity, etc. For testing, a constructor function can be used to make a random fabric, or the fabric can be specified. Then, the evolution over one timestep uses a single function "fabE!" which accepts a Fabric instance as an argument and advances the fabric by one timestep. The "fabE!" function itself is polymorphic and will return based on the specific type supplied to it.

\section{Goals}
\begin{itemize}
\item In the near term, reproduce general small-girdle fabrics by modeling rotation and migration recrystallization. This will show that the model is able to handle recrystallization in arbitrary strain environments.

\item Reproduce WAIS divide thin sections, especially the transition between vertical girdle to single maxima. This may be able to be explained by changing impurity concentrations affecting grain mobility or initial farbic.

\item When the flow model is developed, I will investigate the coupling of the fabric model and the flow. In addition to effects such as boudinage which depend on differing rheologies, it will also be possible to investigate effects due purely to the interdependence of flow and fabric. Recent results \citep{montgomery-smith2011} show that coupled Stokes-Jefferys equations modeling fiber suspensions can induce intiial perturbations in the orientation distribution function to grow significantly in short amounts of time.
\end{itemize}
\bibliographystyle{agu}
\bibliography{anis}
\end{document}
